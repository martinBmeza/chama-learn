\documentclass[11pt]{article}
\usepackage[utf8]{inputenc}	% Para caracteres en español
\usepackage{carlito} %fuente calibri
\usepackage{amsmath,amsthm,amsfonts,amssymb,amscd}
\usepackage{multirow,booktabs}
\usepackage[table]{xcolor}
\usepackage{fullpage}
\usepackage{lastpage}
\usepackage{enumitem}
\usepackage{fancyhdr}
\usepackage{mathrsfs}
\usepackage{wrapfig}
\usepackage{setspace}
\usepackage{calc}
\usepackage{multicol}
\usepackage{cancel}
\usepackage[retainorgcmds]{IEEEtrantools}
\usepackage[margin=3cm]{geometry}
\usepackage{amsmath}
\newlength{\tabcont}
\setlength{\parindent}{0.0in}
\setlength{\parskip}{0.05in}
\usepackage{empheq}
\usepackage{framed}
\usepackage[most]{tcolorbox}
\usepackage{xcolor}
\colorlet{shadecolor}{orange!15}
\parindent 0in
\parskip 12pt
\geometry{margin=1in, headsep=0.25in}


\begin{document}
\setmainfont{Carlito}

\thispagestyle{empty}

\begin{center}
{\LARGE \bf Notas: Redes Neuronales Recurrentes}\\
{\large Martin Bernardo Meza}\\
2020
\end{center}

%Secciones
\input{sections/01-Introduccion.tex}
\section{Arquitecturas Básicas}
\subsection{Red Recurrente simple}

\subsection{Problema de las secuencias largas}
Para entrenar una red recurrente sobre secuencias muy largas, tenemos que ejecutar muchos pasos temporales, lo que hace que se convierta en una red muy profunda (la secuencia al hacer el despliegue se vuelve muy larga). Esto genera problemas de  inestabilidad de gradiente, por lo cual puede tomar mucho tiempo de entrenar o el entrenamiento puede ser inestable. Además, trabajar con secuencias largas hace que al final la red deje de considerar las primeras entradas (las olvida). 
\section{Generación de texto}

Concepto de "natural language" como habilidad cognitiva de los seres humanos que se intenta alcanzar en menor o mayor medida a partir de modelos de machine learning.

Para hacerlo se pueden emplear redes recurrentes. 

Un enfoque es predecir caracteres usando redes RNN

Se pueden armar modelos stateless o statefull. Los modelos stateless aprenden de cachitos aleatorios del texto de entrenamiento, sin realmente considerar una relación entre cada instancia de entrenamiento. Los modelos statefull en cambio, aprenden considerando la estructura global presente en el texto de entrenamiento. 

\subsection{Stateless}
\newpage


\end{document}